\documentclass[twoside,11pt]{article}

% Any additional packages needed should be included after pgm2016.
% Note that pgm2016.sty includes epsfig, amssymb, natbib and graphicx,
% and defines many common macros, such as 'proof' and 'example'.
%
% It also sets the bibliographystyle to plainnat; see the natbib 
% documentation for more informations. 

\usepackage{pgm2016}

% Definitions of handy macros can go here

\newcommand{\dataset}{{\cal D}}
\newcommand{\fracpartial}[2]{\frac{\partial #1}{\partial  #2}}

% Heading arguments are {volume}{year}{pages}{submitted}{published}{author-full-names}

\pgmheading{1}{2000}{1-48}{4/00}{10/00}{Style and Elegance}

% Short headings should be running head and authors last names

\ShortHeadings{Writing a good paper}{Style and Elegance}
\firstpageno{1}

\begin{document}

\title{Writing a good paper}

\author{\name Style \email style@style.st \\
       \addr Department of Applied Style \\
       University of Science \\ 
       Utopia City 
       \AND
       \name Elegance \email elegance@elegance.el \\
       \addr Division of Structural Elegance \\
       University of Science \\ 
       Utopia City}

\editor{Slightly under-indulgent editor}

\maketitle

\begin{abstract}%   <- trailing '%' for backward compatibility of .sty file
Presentation is important, in every aspect of life. 
Moreover when you want to present your revolutionary ideas to the whole
world. Follow the example of this sample paper, and we'll make your work even more awesome. 
\end{abstract}

\begin{keywords}
  Revolutionary ideas, changing the world
\end{keywords}
\section{Introduction}

In this section we present what already knows, yet we have to invent new ways for saying it. 

\subsection{Motivation}

Some useful information for the authors.

\begin{enumerate}
\item Note that {\tt pgm2016.sty} includes {\tt epsfig}, {\tt amssymb}, {\tt natbib} and {\tt graphicx}, and defines many common macros, such as {\tt proof} and {\tt example}. Do not include the package {\tt theorem}, as it is known to conflict with the style.
\item If you need native support to some special characters use UTF8 enconding as follows:
\begin{verbatim}
\usepackage[utf8]{inputenc}
\inputencoding{utf8}
\end{verbatim}
\item Stick to standard tex packages. If you need to define custom commands (for example the common {\tt \\eg} for "e.g.") please do so in your main .tex file.
\item For images use PDF or PNG files, included with {\tt \\includegraphics{imagefile}}. Avoid EPS images and the "epsfig" package, and the "\graphicspath" command.
\end{enumerate}
% Acknowledgements should go at the end, before appendices and references
\acks{We would like to acknowledge support for this project
from the Random Science Foundation (NSF grant IIS-0000000). }

% Manual newpage inserted to improve layout of sample file - not
% needed in general before appendices/bibliography.

\newpage

\appendix
\section*{Appendix A.}
\label{app:theorem}

In this appendix we present a random filler theorem just for reaching the minimum page requirement. 

\noindent
{\bf Theorem} {\it Let $u,v,w$ be discrete variables such that $v, w$ do
not co-occur with $u$ (i.e., $u\neq0\;\Rightarrow \;v=w=0$ in a given
dataset $\dataset$). Let $N_{v0},N_{w0}$ be the number of data points for
which $v=0, w=0$ respectively, and let $I_{uv},I_{uw}$ be the
respective empirical mutual information values based on the sample
$\dataset$. Then
\[
	N_{v0} \;>\; N_{w0}\;\;\Rightarrow\;\;I_{uv} \;\leq\;I_{uw}
\]
with equality only if $u$ is identically 0.} \hfill\BlackBox

\noindent
{\bf Proof}. We use the notation:
\[
P_v(i) \;=\;\frac{N_v^i}{N},\;\;\;i \neq 0;\;\;\;
P_{v0}\;\equiv\;P_v(0)\; = \;1 - \sum_{i\neq 0}P_v(i).
\]
These values represent the (empirical) probabilities of $v$
taking value $i\neq 0$ and 0 respectively.  Entropies will be denoted
by $H$. We aim to show that $\fracpartial{I_{uv}}{P_{v0}} < 0$....\\

{\noindent \em Remainder omitted in this sample. See http://www.jmlr.org/papers/ for full paper.}

\vskip 0.2in
\bibliography{sample}
\end{document}
