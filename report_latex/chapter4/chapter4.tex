
%%%%%%%%%%%%%%%%%%%%%%%%%%%%%%%%%%%%%%%%%%%%%%%%%%%%%%%%%%{
%\documentclass[twoside,11pt]{article}
\documentclass[UTF8]{ctexart}
%%%%% PACKAGES %%%%%%
\usepackage{pgm2016}
\usepackage{amsmath}
\usepackage{algorithm}
\usepackage[noend]{algpseudocode}
\usepackage{subcaption}
\usepackage[english]{babel}	
\usepackage{paralist}	
\usepackage[lowtilde]{url}
\usepackage{fixltx2e}
\usepackage{listings}
\usepackage{color}
\usepackage{hyperref}
\usepackage{mdframed}
\usepackage{amssymb}

%\usepackage{auto-pst-pdf}
\usepackage{pst-all}
\usepackage{pstricks-add}

%%%%% MACROS %%%%%%
\algrenewcommand\Return{\State \algorithmicreturn{} }
\algnewcommand{\LineComment}[1]{\State \(\triangleright\) #1}
\renewcommand{\thesubfigure}{\roman{subfigure}}
\definecolor{codegreen}{rgb}{0,0.6,0}
\definecolor{codegray}{rgb}{0.5,0.5,0.5}
\definecolor{codepurple}{rgb}{0.58,0,0.82}
\definecolor{backcolour}{rgb}{0.95,0.95,0.92}
\lstdefinestyle{mystyle}{
	backgroundcolor=\color{backcolour},  
	commentstyle=\color{codegreen},
	keywordstyle=\color{magenta},
	numberstyle=\tiny\color{codegray},
	stringstyle=\color{codepurple},
	basicstyle=\footnotesize,
	breakatwhitespace=false,        
	breaklines=true,                
	captionpos=b,                    
	keepspaces=true,                
	numbers=left,                    
	numbersep=5pt,                  
	showspaces=false,                
	showstringspaces=false,
	showtabs=false,                  
	tabsize=2
}
\lstset{style=mystyle}

\newenvironment{problem}[2][问题]
{\begin{mdframed}[backgroundcolor=gray!20] \textbf{#1 #2} \\}
	{\end{mdframed}}

\newenvironment{dingyi}[2][定义]
{\begin{mdframed}[backgroundcolor=gray!20] \textbf{#1 #2} \\}
	{\end{mdframed}}

\newenvironment{dingli}[2][定理]
{\begin{mdframed}[backgroundcolor=gray!20] \textbf{#1 #2} \\}
	{\end{mdframed}}

\newcommand{\tabincell}[2]{\begin{tabular}{@{}#1@{}}#2\end{tabular}}
%%%%%%%%%%%%%%%%%%%%%%%%%%%%%%%%%%%%%%%%%%%%%%%%%%%%%%%%%% 


%%%%%%%%%%%%%%%%%%%%Information%%%%%%%%%%%%%%%%%%%%%%%%%%%
\newcommand\course{sd03431280}
\newcommand\courseName{数值计算方法}
\newcommand\semester{2020·春季}
\newcommand\assignmentNumber{4}
\newcommand\studentName{陈路}
\newcommand\studentEmail{chenlu.scien@gmail.com}
\newcommand\studentNumber{201800301206}
\newcommand\addr{泰山学堂2018级计算机取向}
%%%%%%%%%%%%%%%%%%%%%%%%%%%%%%%%%%%%%%%%%%%%%%%%%%%%%%%%%%



%%%%%%%%%%%%%%%%%%%%%%%%%%%%%%%%%%%%%%%%%%%%%%%%%%%%%%%%%%
    \ShortHeadings{山东大学\quad \semester\quad  \course\quad \courseName}{\addr \quad \studentName \quad \studentNumber}
    \firstpageno{1}
    
    \begin{document}
    
    \title{第\assignmentNumber 章\quad 数值微分与数值积分}
    
    \author{\name \studentName \email \studentEmail \\
    \studentNumber
    }
    
    \maketitle
%%%%%%%%%%%%%%%%%%%%%%%%%%%%%%%%%%%%%%%%%%%%%%%%%%%%%%%%%%

\section{分析讨论题}

\begin{problem}{1.1}
	推导复合(Composite)梯形公式及其误差估计.
\end{problem}
\textbf{解}:\\
\begin{dingli}{1}
	复合梯形公式:设等距节点$[x_k,x_{k+1}], k=0,1,\cdots,M$将区间$[a,b]$划分为宽度为
	$h=(b-a)/M$的$M$个子区间$x_k=a+kh$.M个子区间的组合梯形公式
	\begin{align}
		T(f, h)=\frac{h}{2}(f(a)+f(b))+h \sum_{k=1}^{M-\mathrm{r}} f\left(x_{k}\right)\notag
	\end{align}
	是区间$[a,b]$上$f(x)$积分的逼近,即
	\begin{align}
		\int_{a}^{b} f(x) \mathrm{d} x \approx T(f, h) \notag
	\end{align}
\end{dingli}
\textbf{证明}:\\
在每个子区间$[x_{k-1},x_{k}]$上应用梯形公式
\begin{align}
	\int_{x_0}^{x_1} f(x) d x \approx \frac{h}{2} (f(x_0) + f(x_1))
\end{align}
利用子区间上积分的可加性: 
\begin{align}
	\int_{a}^{b} f(x) d x=\sum_{k=1}^{M} \int_{x_{k-1}}^{x_{k}} f(x) d x \approx \sum_{k=1}^{M} \frac{h}{2}\left(f\left(x_{k-1}\right)+f\left(x_{k}\right)\right)\notag
\end{align}
证毕. $\hfill\blacksquare$ 
\begin{dingli}{2}
	复合梯形公式的误差分析:设区间$[a,b]$划分为宽度为$h=(b-a)/M$的$M$个子区间$x_k=a+kh$,组合梯形公式
	\begin{align}
	T(f, h)=\frac{h}{2}(f(a)+f(b))+h \sum_{k=1}^{M-\mathrm{r}} f\left(x_{k}\right)\notag
	\end{align}
	是对积分
	\begin{align}
	\int_{a}^{b} f(x) \mathrm{d} x = T(f, h) + E_{T}(f,h) \notag
	\end{align}
	的逼近. 如果$f\in C^{2}[a,b]$, 则存在值$c, a<c<b$, 使得误差项$E_{T}(f,h)$具有形式
	\begin{align}
	E_{T}(f, h)=\frac{-(b-a) f^{(2)}(c) h^{2}}{12}=O\left(h^{2}\right)\notag
	\end{align}
\end{dingli}
\textbf{证明}:\\
首先确定公式在区间$[x_0,x_1]$上的误差项. 对拉格朗日多项式$P_{1}(x)$及其余项进行积分得:
\begin{align}
	\int_{x_{0}}^{x_{1}} f(x) d x=\int_{x_{0}}^{x_{1}} P_{1}(x) d x+\int_{x_{0}}^{x_{1}} \frac{\left(x-x_{0}\right)\left(x-x_{1}\right) f^{(2)}(c(x))}{2 !} d x \notag
\end{align}
在区间$[x_0,x_1]$上, 项$\left(x-x_{0}\right)\left(x-x_{1}\right)$符号不变, 而$f^{(2)}(c(x))$连续. 因此由积分第二中值定理知: 存在值$c_1$, 使得
\begin{align}
	\int_{x_{0}}^{x_{1}} f(x) d x=\frac{h}{2}\left(f_{0}+f_{1}\right)+f^{(2)}\left(c_{1}\right) \int_{x_{0}}^{x_{1}} \frac{\left(x-x_{0}\right)\left(x-x_{1}\right)}{2 !} d x \notag
\end{align}
对上式右端的积分进行变量代换$x=x_0 + ht$:
\begin{align}
	\begin{aligned}
	\int_{x_{0}}^{x_{1}} f(x) d x &=\frac{h}{2}\left(f_{0}+f_{1}\right)+\frac{f^{(2)}\left(c_{1}\right)}{2} \int_{0}^{1} h(t-0) h(t-1) h d t \\
	&=\frac{h}{2}\left(f_{0}+f_{1}\right)+\frac{f^{(2)}\left(c_{1}\right) h^{3}}{2} \int_{0}^{1}\left(t^{2}-t\right) d t \\
	&=\frac{h}{2}\left(f_{0}+f_{1}\right)-\frac{f^{(2)}\left(c_{1}\right) h^{3}}{12}
	\end{aligned}\notag
\end{align}
将所有区间$[x_{k-1},x_{k}], k=1,2,\cdots,M$上的误差相加,得:
\begin{align}
	\begin{aligned}
	\int_{a}^{b} f(x) d x &=\sum_{k=1}^{M} \int_{x_{k-1}}^{x_{k}} f(x) d x \\
	&=\sum_{k=1}^{M} \frac{h}{2}\left(f\left(x_{k-1}\right)+f\left(x_{k}\right)\right)-\frac{h^{3}}{12} \sum_{k=1}^{M} f^{(2)}\left(c_{k}\right)
	\end{aligned} \notag
\end{align}
第一个和式为复合梯形公式$T(f,h)$. 在第二项中,将一个$h$因子用等式$h=(b-a)/M$替换,得:
\begin{align}
	\int_{a}^{b} f(x) d x = T(f,h) - \frac{(b-a)h^{2}}{12} \left( \frac{1}{M} \sum_{k=1}^{M} f^{(2)}\left(c_{k}\right) \right) \notag
\end{align}
括号中的项可看成是2阶导数的一个均值,因此可用$f^{(2)}(c)$替换,从而得到
\begin{align}
	\int_{a}^{b} f(x) d x = T(f,h) - \frac{(b-a) f^{(2)}(c) h^{2}}{12} \notag
\end{align}
证毕. $\hfill\blacksquare$ 

\begin{problem}{1.2}
	推导基于误差控制的逐次半积分梯形公式及其误差估计.
\end{problem}
\textbf{解}:\\
\begin{dingyi}{1}
	梯形公式序列: 定义$T(0)=(h/2)(f(a)+f(b))$为步长$h=b-a$的梯形公式. 对于所有$J≥1$, 定义$T(J) = T(f,h)$, 其中$T(f,h)$为步长$h=(b-a)/2^{J}$的梯形公式.
\end{dingyi}
\begin{dingli}{3}
	由$T(0)=(h/2)(f(a)+f(b))$开始,梯形公式序列${T(J)}$可由以下递归公式生成:
	\begin{align}
		T(J)=\frac{T(J-1)}{2}+h \sum_{k=1}^{M} f\left(x_{2 k-1}\right), \quad J=1,2, \cdots \notag
	\end{align}
	其中,$h=(b-a)/2^{J}$, $x_k = a + kh, k=0,1,\cdots,M$.
\end{dingli}
\textbf{证明}:\\
对偶数节点$x_{0} < x_{2} < \cdots < x_{2M-2} < x_{2M}$,使用步长为$2h$的梯形公式:
\begin{align}
	T(J-1)=\frac{2 h}{2}\left(f_{0}+2 f_{2}+2 f_{4}+\cdots+2 f_{2 M-4}+2 f_{2 M-2}+f_{2 M}\right) \notag
\end{align}
对全体节点$x_{0} < x_{1} < x_{2} < \cdots < x_{2M-1} < x_{2M}$,使用步长为$h$的梯形公式:
\begin{align}
	T(J)=\frac{h}{2}\left(f_{0}+2 f_{1}+2 f_{2}+\cdots+2 f_{2 M-2}+2 f_{2 M-1}+f_{2 M}\right) \notag
\end{align}
将上式中奇数项和偶数项的和分别累加,得:
\begin{align}
	\begin{aligned}
		T(J) &=\frac{h}{2}\left(f_{0}+2 f_{2}+\cdots+2 f_{2 M-2}+f_{2 M}\right) + h \sum_{k=1}^{M} f_{2k-1} &= \frac{T(J-1)}{2}+h \sum_{k=1}^{M} f\left(x_{2 k-1}\right)
	\end{aligned} \notag
\end{align}
证毕. $\hfill\blacksquare$

\begin{problem}{2}
	令$h=(b-a)/3, x_0 = a, x_1 = a + h, x_2 = b$. 请给出积分公式
	\begin{align}
		\int_{a}^{b} f(x) \mathrm{d} x=\frac{9}{4} h f\left(x_{1}\right)+\frac{3}{4} h f\left(x_{2}\right)\notag
	\end{align}
	的精确度.
\end{problem}
\textbf{解}:\\
由精确度的定义:\\
一个数值积分公式$Q(f)$的精确度是一个最大的正整数$n$,使得
\begin{align}
	\int_{a}^{b} x^{k} \mathrm{d} x=Q(x^{k}), k=0,1,\cdots,n.\notag
\end{align}
分别令$f(x)=x^{0}=1, f(x)=x, f(x)=x^{2}, \cdots$\\
积分得:$\int_{a}^{b} x^{k} \mathrm{d} x=\frac{1}{k+1} (b^{k+1}-a^{k+1}).$\\
计算过程如表\ref{t1}所示.
\begin{table}[htbp]
	\centering
	\caption{积分公式精确度计算过程}
	\label{t1}
	\begin{tabular}{|l|c|l|l|}
		\hline$k$ & $f(x)$ & $\frac{1}{k+1}\left(b^{k+1}-a^{k+1}\right)$ & $\frac{9}{4} h f(a+h)+\frac{3}{4} h f(b)$ \\
		\hline 0 & 1 & $b-a$ & $\frac{9}{4} h+\frac{3}{4} h=3 h=b-a$ \\
		\hline 1 & $x$ & $\frac{1}{2}\left(b^{2}-a^{2}\right)$ & \tabincell{l}{$\frac{9}{4} h(a+h)+\frac{3}{4} h(b)=\frac{3 h}{4}(3(a+h)+b)$ $\frac{b-a}{4}(3 a+b-a+b)$ \\ $=\frac{b-a}{4}(2 a+2 b)=\frac{1}{2} b^{2}-\frac{1}{2} a^{2}$}\\
		\hline 2 & $x^{2}$ & $\frac{1}{3} b^{3}-\frac{1}{3} a^{3}$ & $\frac{9}{4} h(a+h)^{2}+\frac{3}{4} h b^{2}=\frac{1}{3} b^{3}-\frac{1}{3} a^{3}$ \\
		\hline 3 & $x^{3}$ & $\frac{1}{4} b^{4}-\frac{1}{4} a^{4}$ & $\frac{9}{4} h(a+h)^{3}+\frac{3}{4} h b^{3}=-\frac{1}{9} b a^{3}+\frac{1}{6} b^{2} a^{2}-\frac{1}{9} b^{3} a+\frac{5}{18} b^{4}-\frac{2}{9} a^{4}$ \\
		\hline
	\end{tabular}
\end{table}
\\可见,积分公式$\int_{a}^{b} f(x) \mathrm{d} x=\frac{9}{4} h f\left(x_{1}\right)+\frac{3}{4} h f\left(x_{2}\right)$的精确度为2.
\section{上机实验题}

\begin{problem}{3.1}
	自行编制复合梯形公式、Simpson公式的计算程序.
\end{problem}
\textbf{复合梯形公式的代码实现}
\begin{lstlisting}[language=matlab]
function s = traprl(f, a, b, M)
% Composite Trapezoidal Rule
% Input  - f   the integrand input as string 'f'
%        - a,b upper and lower limits of integration
%        - M   the number of subintervals
% Output - s   the sum of Trapezoidal Rule

h = (b-a)/M;
s = 0;

for k = 1 : (M-1)
	x = a + h*k;
	s = s + feval(f,x);
end
s = h*(feval(f,a) + feval(f,b))/2 + h*s;

end
\end{lstlisting}
\textbf{复合Simpson公式的代码实现}
\begin{lstlisting}[language=matlab]
function s = simprl(f, a, b, M)
% Composite Simpson Rule
% Input  - f   the integrand input as string 'f'
%        - a,b upper and lower limits of integration
%        - M   the number of subintervals
% Output - s   the sum of Trapezoidal Rule

h = (b-a)/(2*M);
s1 = 0;
s2 = 0;

for k = 1 : M
	x = a + h*(2*k-1);
	s1 = s1 + feval(f,x);
end
for k = 1 : (M-1)
	x = a + h*(2*k);
	s2 = s2 + feval(f,x);
end
s = h*(feval(f,a) + feval(f,b) + 4*s1 + 2*s2)/3;

end
\end{lstlisting}

\begin{problem}{3.2}
	取$h=0.01$,分别利用复合梯形公式、Simpson公式计算定积分
	\begin{align}
		I(f)=\frac{1}{\sqrt{2 \pi}} \int_{0}^{1} \exp {-\frac{x^{2}}{2}} \mathrm{d} x\notag
	\end{align}
	试与精确解比较,说明两种格式的优劣.
\end{problem}
\textbf{解}:\\
\begin{lstlisting}[language=matlab]
function y = f(x)
y = (1/sqrt(2*pi)) * exp(-(x^2)/2);
\end{lstlisting}
\begin{lstlisting}[language=matlab]
syms x;
t = traprl('f',0,1,100);
s = simprl('f',0,1,100);
\end{lstlisting}
运行上述程序,得:\\
复合梯形公式求解结果$t = 0.341342729639117$, 
复合Simpson公式求解结果$s = 0.341344746070223$,
精确值$r = 0.341344746068543$.

\begin{problem}{3.3}
	若取计算精度为$10^{-4}$,则$h = ?$,$n = ?$.
\end{problem}
\textbf{解}:\\
\textbf{I}. 考虑使用复合梯形公式的情况\\
被积函数为$f(x)= \exp (-\frac{x^{2}}{2})$,其前两阶导数为
\begin{align}
	f'(x) = -x \cdot \exp (-\frac{x^{2}}{2});
	f''(x) = (x^2 - 1) \exp (-\frac{x^{2}}{2}). \notag
\end{align}
区间$[0,1]$上$f''(x)$单调递增,$f''(0)=-1, f''(1)=0$,$|f''(x)|$的最大值在端点$x=0$处取得,从而对$0\leq c \leq 1$,$|f''(c)| \leq |f''(0)| = 1$.\\
代入复合梯形公式误差项得
\begin{align}
	|E_{T}(f, h)| = \frac{|-(b-a) f^{(2)}(c) h^{2}|}{12} \leq \frac{h^2}{12}. \notag
\end{align}
步长$h$与$n$满足关系$h=1/n$,代入上式,得:
\begin{align}
	|E_{T}(f, h)| \leq \frac{1}{12\cdot n^2} \leq 10^{-4}. \notag
\end{align}
解得:$n>28.8675$. \\
由于$n$必须为整数,取$n=29$,对应的步长$h=1/n\approx 0.0344827586$.\\
\textbf{II}. 考虑使用复合Simpson公式的情况\\
被积函数为$f(x)= \exp (-\frac{x^{2}}{2})$,其四阶导数为
\begin{align}
	f^{(4)}(x) = (x^4 - 6x^2 + 3) \exp (-\frac{x^{2}}{2}). \notag
\end{align}
区间$[0,1]$上$f^{(4)}(x)$单调递减,$f^{(4)}(0)=3, f^{(4)}(1)=-2/\sqrt{e}$,$|f^{(4)}(x)|$的最大值在端点$x=0$处取得,从而对$0\leq c \leq 1$,$|f^{(4)}(c)| \leq |f^{(4)}(0)| = 3$.\\
代入复合Simpson公式误差项得
\begin{align}
	|E_{T}(f, h)| = \frac{|-(b-a) f^{(4)}(c) h^{4}|}{180} \leq \frac{3h^4}{180} = \frac{h^4}{60}. \notag
\end{align}
步长$h$与$n$满足关系$h=1/n$,代入上式,得:
\begin{align}
	|E_{T}(f, h)| \leq \frac{1}{60\cdot n^4} \leq 10^{-4}. \notag
\end{align}
解得:$n>3.5930$. \\
由于$n$必须为整数,取$n=4$,对应的步长$h=1/n\approx 0.25$.

\end{document}
