
%%%%%%%%%%%%%%%%%%%%%%%%%%%%%%%%%%%%%%%%%%%%%%%%%%%%%%%%%%{
%\documentclass[twoside,11pt]{article}
\documentclass[UTF8]{ctexart}
%%%%% PACKAGES %%%%%%
\usepackage{pgm2016}
\usepackage{amsmath}
\usepackage{algorithm}
\usepackage[noend]{algpseudocode}
\usepackage{subcaption}
\usepackage[english]{babel}	
\usepackage{paralist}	
\usepackage[lowtilde]{url}
\usepackage{fixltx2e}
\usepackage{listings}
\usepackage{color}
\usepackage{hyperref}
\usepackage{mdframed}

%\usepackage{auto-pst-pdf}
\usepackage{pst-all}
\usepackage{pstricks-add}

%%%%% MACROS %%%%%%
\algrenewcommand\Return{\State \algorithmicreturn{} }
\algnewcommand{\LineComment}[1]{\State \(\triangleright\) #1}
\renewcommand{\thesubfigure}{\roman{subfigure}}
\definecolor{codegreen}{rgb}{0,0.6,0}
\definecolor{codegray}{rgb}{0.5,0.5,0.5}
\definecolor{codepurple}{rgb}{0.58,0,0.82}
\definecolor{backcolour}{rgb}{0.95,0.95,0.92}
\lstdefinestyle{mystyle}{
	backgroundcolor=\color{backcolour},  
	commentstyle=\color{codegreen},
	keywordstyle=\color{magenta},
	numberstyle=\tiny\color{codegray},
	stringstyle=\color{codepurple},
	basicstyle=\footnotesize,
	breakatwhitespace=false,        
	breaklines=true,                
	captionpos=b,                    
	keepspaces=true,                
	numbers=left,                    
	numbersep=5pt,                  
	showspaces=false,                
	showstringspaces=false,
	showtabs=false,                  
	tabsize=2
}
\lstset{style=mystyle}

\newenvironment{problem}[2][问题]
{\begin{mdframed}[backgroundcolor=gray!20] \textbf{#1 #2} \\}
	{\end{mdframed}}

\newenvironment{dingyi}[2][定义]
{\begin{mdframed}[backgroundcolor=gray!20] \textbf{#1 #2} \\}
	{\end{mdframed}}

\newenvironment{dingli}[2][定理]
{\begin{mdframed}[backgroundcolor=gray!20] \textbf{#1 #2} \\}
	{\end{mdframed}}

%%%%%%%%%%%%%%%%%%%%%%%%%%%%%%%%%%%%%%%%%%%%%%%%%%%%%%%%%% 


%%%%%%%%%%%%%%%%%%%%Information%%%%%%%%%%%%%%%%%%%%%%%%%%%
\newcommand\course{sd03431280}
\newcommand\courseName{数值计算方法}
\newcommand\semester{2020·春季}
\newcommand\assignmentNumber{9}
\newcommand\studentName{陈路}
\newcommand\studentEmail{chenlu.scien@gmail.com}
\newcommand\studentNumber{201800301206}
\newcommand\addr{泰山学堂2018级计算机取向}
%%%%%%%%%%%%%%%%%%%%%%%%%%%%%%%%%%%%%%%%%%%%%%%%%%%%%%%%%%



%%%%%%%%%%%%%%%%%%%%%%%%%%%%%%%%%%%%%%%%%%%%%%%%%%%%%%%%%%
    \ShortHeadings{山东大学\quad \semester\quad  \course\quad \courseName}{\addr \quad \studentName \quad \studentNumber}
    \firstpageno{1}
    
    \begin{document}
    
    \title{第\assignmentNumber 章\quad 特征值与特征向量}
    
    \author{\name \studentName \email \studentEmail \\
    \studentNumber
    }
    
    \maketitle
%%%%%%%%%%%%%%%%%%%%%%%%%%%%%%%%%%%%%%%%%%%%%%%%%%%%%%%%%%
本文首先简要总结特征值与特征向量的基础知识,给出相关的定义和定理。进而对求解特征值的幂方法和QR方法进行代码实现,并结合具体题目运行程序,展示不同算法的结果。

\section{特征值与特征向量基础知识}
\subsection{特征值与特征向量定义}
\begin{dingyi}{1}
	如果A是一个$n\times n$实矩阵,则它的n个\textbf{特征值}$\lambda_1, \lambda_2, \cdots, \lambda_n$是下面$n$阶特征多项式的实根或复根:
	\begin{align}
		p(\lambda)=det(\mathbf{A} - \lambda \mathbf{I}). \notag
	\end{align}
\end{dingyi}
\begin{dingyi}{2}
	如果$\lambda$是$\mathbf{A}$的特征值并且非零向量$\mathbf{V}$具有如下特性:
	\begin{align}
	\mathbf{AV} = \lambda \mathbf{V}. \notag
	\end{align}
\end{dingyi}
\subsection{特征值范围估计}
\begin{dingyi}{3}
	设$\|X\|$是向量范数,则对应的\textbf{自然矩阵范数}为
	\begin{align}
		\|A\|=\max _{\|X\|=1}\left\{\frac{\|A X\|}{\|X\|}\right\} \notag
	\end{align}
	范数$\|A\|_{\infty}$可表示为
	\begin{align}
		\|A\|_{\infty}=\max _{1\leq i\leq n}\left\{\sum_{j=1}^{n}|a_{ij}|\right\} \notag
	\end{align}
\end{dingyi}
\begin{dingli}{1}
	如果$\lambda$是矩阵$\mathbf{A}$的任意特征值,则对于所有自然矩阵范数$\|A\|$,有
	\begin{align}
		|\lambda| \leq \|A\| \notag
	\end{align}
\end{dingli}
\begin{dingli}{2}
	\textbf{Gerschgorin圆盘定理}:\\
	设$A$是一个$n\times n$矩阵,$C_j$表示位于复平面$z=x+iy$上,以$a_{jj}$为圆心,以
	\begin{align}
		r_j = \sum_{k=1,k\neq j}^{n} |a_{jk}|, \quad j=1,2,\cdots,n \notag
	\end{align}
	为半径的圆盘,即$C_j$包含所有满足条件
	\begin{align}
			C_j = \{z:|z-a_{jj}|\leq r_j\} \notag
	\end{align}
	的复数$z=x+iy$.\\
	如果$S=\cup^{n}_{i=1} C_i$,则$A$的所有特征值包含在集合$S$中.\\
	进一步可得,以上$k$个圆盘的并如果与其余的$n-k$个圆盘不相交,则它们一定包含$k$个特征值(含重复).
\end{dingli}
\begin{dingli}{3}
	\textbf{谱半径定理}:\\
	设$A$是一个对称阵,则$A$的谱半径定义为$\|A\|_{2}$,并且满足如下关系
	\begin{align}
		\|A\|_{2} = \max{|\lambda_1|, |\lambda_2|, \cdots, |\lambda_n|}. \notag
	\end{align}
\end{dingli}
\section{上机实验题}

\begin{problem}{1}
	已知矩阵
	\begin{align}
		\mathbf{A}=\left[\begin{array}{ccc}
		4 & -1 & 1 \\
		-1 & 3 & -2 \\
		1 & -2 & 3
		\end{array}\right]\notag
	\end{align}
	是一个对称矩阵,且其特征值为$\lambda_1 = 6$, $\lambda_2 = 3$, $\lambda_3 = 1$.\\
	分别利用幂法、对称幂法、反幂法求其最大特征值和特征向量.\\
	注意:可取初始向量$\textbf{x}^{(0)}=(1\quad 1\quad 1)^{T}$.
\end{problem}
\textbf{解}:\\
(1)幂法\\
代码实现如下:
\begin{lstlisting}[language=matlab]
function [lambda, V] = power1 (A, X, epsilon, max1)
% Input  - A   is an nxn matrix
%        - X   is the nx1 starting vector
%        - epsilon   is the tolerance
%        - max1 is the maximum number of iterations
% Output - lambda is the dominant eigenvalue
%        - V is the dominant eigenvector

% Initialization
lambda = 0 ;
cnt = 0 ;
err = 1;
state = 1 ;

while ((cnt<=max1) && (state==1))
	Y = A*X;
	%Normalize Y
	[m j] = max(abs(Y));
	c1 = m;
	dc = abs(lambda - c1) ;
	Y = (1/c1)*Y;
	%Update X and lambda and check for convergence
	dv = norm(X- Y) ;
	err = max(dc,dv) ;
	X = Y;
	lambda = c1;
	state = 0;
	if (err > epsilon)
		state = 1;
	end
	cnt = cnt + 1;
end
V = X;
end
\end{lstlisting}
运行程序
\begin{lstlisting}[language=matlab]
syms A X epsilon max1;
A = [4 -1 1;-1 3 -2;1 -2 3];
X = [1 1 1]';
epsilon = 0.0000000001;
max1 = 100;
[lambda, V] = power1 (A, X, epsilon, max1)
\end{lstlisting}
求得:\\
最大特征值为$5.999999999912689$,对应的特征向量为
$$V=(1.000000000000000 \quad -0.999999999978172 \quad 0.999999999978172)^{T}$$
(2)对称幂法\\
代码实现如下:
\begin{lstlisting}[language=matlab]
function [lambda,V] = sympower(A, X, epsilon, max1)
% Input  - A   is an nxn matrix
%        - X   is the nx1 starting vector
%        - epsilon   is the tolerance
%        - max1 is the maximum number of iterations
% Output - lambda is the dominant eigenvalue
%        - V is the dominant eigenvector

X = X/norm(X,2);
for k = 2:max1
	Y = A*X;
	u = X'*Y;
	if norm(Y,2) == 0
		break, end
	err = norm((X-Y/norm(Y,2)),2);
	X = Y/norm(Y,2);
	if(err < epsilon)
		break, end
end
lambda = u;
V = X;
end
\end{lstlisting}
求得:\\
最大特征值为$5.999999999999999$,对应的特征向量为
$$V=(0.577350269256838 \quad -0.577350269156020 \quad 0.577350269156020)^{T}$$
(3)反幂法\\
代码实现如下:
\begin{lstlisting}[language=matlab]
function [lambda,V] = invpower(A, X, alpha, epsilon, max1)
% Input  - A   is an nxn matrix
%        - X   is the nx1 starting vector
%        - alpha    is the given shift
%        - epsilon  is the tolerance
%        - max1 is the maximum number of iterations
% Output - lambda is the dominant eigenvalue
%        - V is the dominant eigenvector

% Initilization
[n n] = size(A);
A = A - alpha * eye(n);
lambda = 0;
cnt = 0;
err = 1;
state = 1;

while ((cnt <= max1) && (state == 1))
	% solve system AY=X
	Y = A\X;
	% normalizae Y
	[m j] = max(abs(Y));
	c1 = m;
	dc = abs(lambda - c1);
	Y = (1/c1)*Y;
	% update X and lambda and check for convergence
	dv = norm(X - Y);
	err = max(dc, dv);
	X = Y;
	lambda = c1;
	state = 0;
	if (err > epsilon)
		state = 1;
	end
	cnt = cnt+1;
end
lambda = alpha + 1/c1;
V = X;
end
\end{lstlisting}
运行程序
\begin{lstlisting}[language=matlab]
syms A X alpha epsilon max1;
A = [4 -1 1;-1 3 -2;1 -2 3];
X = [1 1 1]';
alpha = 5.8;
epsilon = 0.0000000001;
max1 = 100;
[lambda, V] = power1 (A, X, alpha, epsilon, max1)
\end{lstlisting}
求得:\\
最大特征值为$6.000000000000043$,对应的特征向量为
$$V=(1.000000000000000 \quad -0.999999999999947 \quad 0.999999999999947)^{T}$$


\begin{problem}{2}
	分别利用Householder变换和Givens旋转变化方法求\textbf{A}的\textbf{QR}分解
	\begin{align}
		\mathbf{A}=\left[\begin{array}{ccc}
		1 & 0 & 0 \\
		1 & 1 & 0 \\
		1 & 1 & 1 \\
		1 & 1 & 1
		\end{array}\right]\notag
	\end{align}
	写出每一步具体求解过程,及最终分解结果.
\end{problem}
\textbf{解}:\\

1. 使用Householder变换求矩阵$A$的QR分解:\\
\begin{align}
\mathbf{A^{(1)}}=\mathbf{A}=\left[\begin{array}{ccc}
1 & 0 & 0 \\
1 & 1 & 0 \\
1 & 1 & 1 \\
1 & 1 & 1
\end{array}\right]\notag
\end{align}
Step1: 对于$A$的第一列列向量
\begin{align}
	\mathbf{x}_{1}=\mathbf{a}_{1: 4,1}^{(1)}=\left[\begin{array}{c}
	1 \\
	1 \\
	1 \\
	1
	\end{array}\right], \quad\left\|\mathbf{x}_{1}\right\|_{2}=2 \notag
\end{align}
对应的Householder向量为
\begin{align}
	\begin{array}{l}
	\qquad \tilde{\mathbf{v}}_{1}=\mathbf{x}_{1}+\left\|\mathbf{x}_{1}\right\|_{2} \mathbf{e}_{1}=\left[\begin{array}{c}
	1 \\
	1 \\
	1 \\
	1 
	\end{array}\right]+2\left[\begin{array}{r}
	1 \\
	0 \\
	0 \\
	0 
	\end{array}\right]=\left[\begin{array}{c}
	3 \\
	1 \\
	1 \\
	1 
	\end{array}\right]
	\end{array} \notag
\end{align}
借助此向量,构造Householder反射
\begin{align}
	c=\frac{2}{\tilde{\mathbf{v}}_{1}^{T} \tilde{\mathbf{v}}_{1}}=1/6, \quad \tilde{H}_{1}=I-c \tilde{\mathbf{v}}_{1} \tilde{\mathbf{v}}_{1}^{T} \notag
\end{align}
将此反射作用于$A^{(1)}$,得
\begin{align}
	A^{(2)} = \tilde{H}_{1} A^{(1)} = 
	\left[\begin{array}{ccc}
	2 &	1.5 &	1 \\
	0 &	-0.866025403784438 &	-0.577350269189625 \\
	0 &	1.11022302462516e-16 &	-0.816496580927726 \\
	0 &	1.11022302462516e-16 &	1.11022302462516e-16
	\end{array}\right]  \notag
\end{align}
Step2: 对于$A^{(2)}$的第二列列向量2至4行的分量
\begin{align}
\mathbf{x}_{2}=\mathbf{a}_{2: 4,2}^{(2)}=\left[\begin{array}{c}
-0.866025403784438 \\
1.11022302462516e-16 \\
1.11022302462516e-16
\end{array}\right], \quad\left\|\mathbf{x}_{2}\right\|_{2}=0.866025403784438 \notag
\end{align}
对应的Householder向量为
\begin{align}
\begin{aligned}
	\qquad \tilde{\mathbf{v}}_{2}=\mathbf{x}_{2}+\left\|\mathbf{x}_{2}\right\|_{2} \mathbf{e}_{1} &=\left[\begin{array}{c}
	-0.866025403784438 \\
	1.11022302462516e-16 \\
	1.11022302462516e-16
	\end{array}\right]+0.866025403784438\left[\begin{array}{r}
	1 \\
	0 \\
	0 
	\end{array}\right]\\
	&=\left[\begin{array}{c}
	0 \\
	1.11022302462516e-16 \\
	1.11022302462516e-16
	\end{array}\right] \notag
\end{aligned}	
\end{align}
借助此向量,构造Householder反射
\begin{align}
c=\frac{2}{\tilde{\mathbf{v}}_{2}^{T} \tilde{\mathbf{v}}_{2}}=8.112963841460618e+31, \quad \tilde{H}_{2}=I-c \tilde{\mathbf{v}}_{2} \tilde{\mathbf{v}}_{2}^{T} \notag
\end{align}
将此反射作用于$A^{(2)}$,得
\begin{align}
\begin{aligned}
	\tilde{H}_{2} A^{(2)} &= 
	\left[\begin{array}{cc}
	0.866025403784438 &	0.577350269189625 \\
	0 &	-0.816496580927726 \\
	0 &	3.70074341541719e-17
	\end{array}\right] \notag \\
	A^{(3)} &=  
	\left[\begin{array}{ccc}
	2.00000000000000 &	1.50000000000000 &	1.00000000000000 \\
	0 &	0.866025403784438 &	0.577350269189625 \\
	0 &	0 &	-0.816496580927726 \\
	0 &	0 &	3.70074341541719e-17
	\end{array}\right].\notag
\end{aligned}
\end{align}
Step3: 对于$A^{(3)}$的第三列列向量3至4行的分量
\begin{align}
\mathbf{x}_{3}=\mathbf{a}_{3: 4,3}^{(3)}=\left[\begin{array}{c}
-0.816496580927726 \\
3.70074341541719e-17
\end{array}\right], \quad\left\|\mathbf{x}_{3}\right\|_{2}=0.816496580927726 \notag
\end{align}
对应的Householder向量为
\begin{align}
\begin{array}{l}
\qquad \tilde{\mathbf{v}}_{3}=\mathbf{x}_{3}+\left\|\mathbf{x}_{3}\right\|_{2} \mathbf{e}_{1}=\left[\begin{array}{c}
-0.816496580927726 \\
3.70074341541719e-17
\end{array}\right]+0.816496580927726\left[\begin{array}{r}
1 \\
0 
\end{array}\right]=\left[\begin{array}{c}
0 \\
0
\end{array}\right]
\end{array} \notag
\end{align}
借助此向量,构造Householder反射
\begin{align}
c=\frac{2}{\tilde{\mathbf{v}}_{3}^{T} \tilde{\mathbf{v}}_{3}}=1/6, \quad \tilde{H}_{3}=I-c \tilde{\mathbf{v}}_{3} \tilde{\mathbf{v}}_{3}^{T} \notag
\end{align}
将此反射作用于$A^{(3)}$,得
\begin{align}
\begin{aligned}
\tilde{H}_{3} A^{(3)} &= 
\left[\begin{array}{c}
0.816496580927726 \\
0
\end{array}\right],\\
R = A^{(4)} &=  
\left[\begin{array}{ccc}
2.00000000000000 &	1.50000000000000 &	1.00000000000000 \\
0 &	0.866025403784438 &	0.577350269189625 \\
0 &	0 &	0.816496580927726 \\
0 &	0 &	0
\end{array}\right].\notag
\end{aligned}
\end{align}
将以上的Householder变换矩阵顺序相乘,得到正交矩阵$Q$
\begin{align}
	Q = H_{1}H_{2}H_{3} = 
	\left[\begin{array}{cccc}
	-0.500000000000000 &  0.866025403784439 &                  0 &  0.000000000000000 \\
	-0.500000000000000 & -0.288675134594813 &  0.816496580927726 &  0.000000000000000 \\
	-0.500000000000000 & -0.288675134594813 & -0.408248290463863 & -0.707106781186548 \\
	-0.500000000000000 & -0.288675134594813 & -0.408248290463863 &  0.707106781186547
	\end{array}\right].\notag
\end{align}
Householder QR分解的代码实现如下:
\begin{lstlisting}[language=matlab]
function [W,A] = house(A)
% Householder QR Factorization
% Input  - A  a symmetric matrix can be factorized as Q*R
% Output - W  NOT the factorized orthogonal matrix Q!
%        - A  the upper triangle matrix

[m,n] = size(A);
W = zeros(m, n);
for k = 1:n
	x = A(k:m, k);
	v = x;
	v(1) = sign(x(1))*norm(x) + v(1);
	v = v/norm(v);
	A(k:m,k:n) = A(k:m,k:n) - 2*v* (v'*A(k:m,k:n));
	W(k:m, k) = v;
end

end
\end{lstlisting}
\begin{lstlisting}[language=matlab]
function Q = formQ (W)
[m,n] = size(W);
Q = eye(m);
for k = n:-1:1
	v = W(k:m,k);
	Q(k:m, :) = Q(k:m, :) - 2*v* (v'*Q(k:m, :));
end
\end{lstlisting}

2. 使用Givens旋转变化求矩阵$A$的QR分解:\\
Step1: 计算一个旋转变化矩阵,使得当其作用于$a_{31}$和$a_{41}$时能够使$a_{41}=0$
\begin{align}
	\left[\begin{array}{cc}
	0.707106781186547 & -0.707106781186547 \\
	0.707106781186547 & 0.707106781186547 
	\end{array}\right]^{T}
	\left[\begin{array}{c}
	1 \\ 1
	\end{array}\right]
	=
	\left[\begin{array}{c}
	1.414213562373095 \\ 0
	\end{array}\right] \notag
\end{align}
将此变换作用于第3和第4行,得:
\begin{align}
	\left[\begin{array}{cccc}
	1 & 0 & 0 & 0 \\
	0 & 1 & 0 & 0 \\
	0 & 0 & 0.707106781186547 & -0.707106781186547 \\
	0 & 0 & 0.707106781186547 & 0.707106781186547 
	\end{array}\right]^{T}
	\left[\begin{array}{ccc}
	1 & 0 & 0 \\
	1 & 1 & 0 \\
	1 & 1 & 1 \\
	1 & 1 & 1
	\end{array}\right] \notag\\
	=
	\left[\begin{array}{ccc}
	1 & 0 & 0 \\
	1 & 1 & 0 \\
	1.414213562373094 & 1.414213562373094 & 1.414213562373094 \\
	0 & 0 & 0
	\end{array}\right] \notag
\end{align}
(由于页面宽度限制,也为了方便阅读,以下步骤中的数据均进行了数值精度的舍弃. 注意只是在此书面报告中舍弃了部分数位,而原程序中未舍弃)\\
Step2: 计算一个旋转变化矩阵,使得当其作用于$a_{21}$和$a_{31}$时能够使$a_{31}=0$
\begin{align}
	\left[\begin{array}{cc}
	0.8165 & -0.5774 \\
	0.5774 & 0.8165 
	\end{array}\right]^{T}
	\left[\begin{array}{c}
	1 \\
	1.4142
	\end{array}\right]
	=
	\left[\begin{array}{c}
	1.7321 \\ 0
	\end{array}\right] \notag
\end{align}
将此变换作用于第2和第3行,得:
\begin{align}
	\left[\begin{array}{cccc}
	1 & 0 & 0 & 0 \\
	0 & 0.5774 & -0.8165 & 0 \\
	0 & 0.8165 & 0.5774 & 0 \\
	0 & 0 & 0 & 1 
	\end{array}\right]^{T}
	\left[\begin{array}{ccc}
	1 & 0 & 0 \\
	1 & 1 & 0 \\
	1.4142 & 1.4142 & 1.4142 \\
	0 & 0 & 0
	\end{array}\right] \notag\\
	=
	\left[\begin{array}{ccc}
	1 & 0 & 0 \\
	1.7321 & 1.7321 & 1.1547 \\
	-1.1102e-16 & -1.1102e-16 & 0.8165 \\
	0 & 0 & 0
	\end{array}\right] \notag
\end{align}
Step3: 计算一个旋转变化矩阵,使得当其作用于$a_{11}$和$a_{21}$时能够使$a_{21}=0$,将此变换作用于第2和第3行,得:
\begin{align}
\left[\begin{array}{cccc}
0.5000 & -0.8660 & 0 & 0 \\
0.8660 & 0.50000 & 0 & 0 \\
0 & 0 & 1 & 0 \\
0 & 0 & 0 & 1 
\end{array}\right]^{T}
\left[\begin{array}{ccc}
1 & 0 & 0 \\
1.7321 & 1.7321 & 1.1547 \\
-1.1102e-16 & -1.1102e-16 & 0.8165 \\
0 & 0 & 0
\end{array}\right] \notag\\
=
\left[\begin{array}{ccc}
2 & 1.5 & 1 \\
-1.1102e-16 & 0.8660 & 0.5774 \\
-1.1102e-16 & -1.1102e-16 & 0.8165 \\
0 & 0 & 0
\end{array}\right] \notag
\end{align}
Step4: $\cdots \cdots$ (原理与前面步骤相同,不再赘述)\\
最终结果:
\begin{align}
\begin{aligned}
	\mathbf{Q} &=
	\left[\begin{array}{cccc}
	0.5 & -0.866025403784439 & -0.000000000000000 &                  0\\
	0.5 &  0.288675134594813 & -0.816496580927726 &                  0\\
	0.5 &  0.288675134594813 &  0.408248290463863 & -0.707106781186547\\
	0.5 &  0.288675134594813 &  0.408248290463863 &  0.707106781186547
	\end{array}\right] \notag\\
	\mathbf{R} &=
	\left[\begin{array}{cccc}
	2.000000000000000 & 1.500000000000000 &  1.000000000000000\\
	-0.000000000000000 &  0.866025403784439 &  0.577350269189626\\
	-0.000000000000000        &           0  & 0.816496580927726\\
	0                 &  0             &      0
	\end{array}\right] \notag
\end{aligned}
\end{align}
Givens QR分解的代码实现如下:
\begin{lstlisting}[language=matlab]
	function [Q,R] = qrgivens(A)
	% Givens QR Factorization
	% Input  - A  a symmetric matrix can be factorized as Q*R
	% Output - Q  the orthogonal matrix
	%        - R  the upper triangle matrix
	
	[m,n] = size(A);
	Q = eye(m);
	R = A;
	
	for j = 1:n
	for i = m:-1:(j+1)
		G = eye(m);
		[c,s] = givensrotation( R(i-1,j),R(i,j) );
		G([i-1, i],[i-1, i]) = [c -s; s c];
		R = G'*R;
		Q = Q*G;
	end
	end
	
	end
\end{lstlisting}
\begin{lstlisting}[language=matlab]
	function [c,s] = givensrotation(a,b)
	
	if b == 0
		c = 1;
		s = 0;
	else
		if abs(b) > abs(a)
			r = a / b;
			s = 1 / sqrt(1 + r^2);
			c = s*r;
		else
			r = b / a;
			c = 1 / sqrt(1 + r^2);
			s = c*r;
		end
	end
	
	end
\end{lstlisting}
\end{document}
