
%%%%%%%%%%%%%%%%%%%%%%%%%%%%%%%%%%%%%%%%%%%%%%%%%%%%%%%%%%{
%\documentclass[twoside,11pt]{article}
\documentclass[UTF8]{ctexart}
%%%%% PACKAGES %%%%%%
\usepackage{pgm2016}
\usepackage{amsmath}
\usepackage{algorithm}
\usepackage[noend]{algpseudocode}
\usepackage{subcaption}
\usepackage[english]{babel}	
\usepackage{paralist}	
\usepackage[lowtilde]{url}
\usepackage{fixltx2e}
\usepackage{listings}
\usepackage{color}
\usepackage{hyperref}
\usepackage{mdframed}

%\usepackage{auto-pst-pdf}
\usepackage{pst-all}
\usepackage{pstricks-add}

%%%%% MACROS %%%%%%
\algrenewcommand\Return{\State \algorithmicreturn{} }
\algnewcommand{\LineComment}[1]{\State \(\triangleright\) #1}
\renewcommand{\thesubfigure}{\roman{subfigure}}
\definecolor{codegreen}{rgb}{0,0.6,0}
\definecolor{codegray}{rgb}{0.5,0.5,0.5}
\definecolor{codepurple}{rgb}{0.58,0,0.82}
\definecolor{backcolour}{rgb}{0.95,0.95,0.92}
\lstdefinestyle{mystyle}{
	backgroundcolor=\color{backcolour},  
	commentstyle=\color{codegreen},
	keywordstyle=\color{magenta},
	numberstyle=\tiny\color{codegray},
	stringstyle=\color{codepurple},
	basicstyle=\footnotesize,
	breakatwhitespace=false,        
	breaklines=true,                
	captionpos=b,                    
	keepspaces=true,                
	numbers=left,                    
	numbersep=5pt,                  
	showspaces=false,                
	showstringspaces=false,
	showtabs=false,                  
	tabsize=2
}
\lstset{style=mystyle}

\newenvironment{problem}[2][问题]
{\begin{mdframed}[backgroundcolor=gray!20] \textbf{#1 #2} \\}
	{\end{mdframed}}

\newenvironment{dingyi}[2][定义]
{\begin{mdframed}[backgroundcolor=gray!20] \textbf{#1 #2} \\}
	{\end{mdframed}}

\newenvironment{dingli}[2][定理]
{\begin{mdframed}[backgroundcolor=gray!20] \textbf{#1 #2} \\}
	{\end{mdframed}}

%%%%%%%%%%%%%%%%%%%%%%%%%%%%%%%%%%%%%%%%%%%%%%%%%%%%%%%%%% 


%%%%%%%%%%%%%%%%%%%%Information%%%%%%%%%%%%%%%%%%%%%%%%%%%
\newcommand\course{sd03431280}
\newcommand\courseName{数值计算方法}
\newcommand\semester{2020·春季}
\newcommand\assignmentNumber{6、7}
\newcommand\studentName{陈路}
\newcommand\studentEmail{chenlu.scien@gmail.com}
\newcommand\studentNumber{201800301206}
\newcommand\addr{泰山学堂2018级计算机取向}
%%%%%%%%%%%%%%%%%%%%%%%%%%%%%%%%%%%%%%%%%%%%%%%%%%%%%%%%%%



%%%%%%%%%%%%%%%%%%%%%%%%%%%%%%%%%%%%%%%%%%%%%%%%%%%%%%%%%%
    \ShortHeadings{山东大学\quad \semester\quad  \course\quad \courseName}{\addr \quad \studentName \quad \studentNumber}
    \firstpageno{1}
    
    \begin{document}
    \title{第\assignmentNumber 章\quad 线性方程组求解}
    \author{\name \studentName \email \studentEmail \\
    \studentNumber
    }\maketitle
%%%%%%%%%%%%%%%%%%%%%%%%%%%%%%%%%%%%%%%%%%%%%%%%%%%%%%%%%%
\section{上机实验题}
\begin{problem}{1}
	求解线性方程组
	\begin{equation}
		\begin{cases}
			4x-y+z = 7 \\
			4x-8y+z = -21 \\
			-2x+y+5z = 15\label{eq:1} 
		\end{cases}
	\end{equation}
	(1)试用LU分解求解此方程组\\
	(2)分别用Jacobi, Gauss-Seidel方法求解此方程组
\end{problem}
\textbf{解}:\\
(1)利用高斯消去法构造方程组\ref{eq:1}的系数矩阵\textbf{A}的三角分解\\
通过将单位矩阵放在\textbf{A}的左边来构造矩阵\textbf{L}.
\begin{equation}
	\begin{aligned}
	A &=
	\left(
		\begin{array}{ccc}
			4 & -1 & 1\\
			4 & -8 & 1\\
			-2 & 1 & 5 
		\end{array}
	\right)\\
	& =
	\left(
		\begin{array}{ccc}
			1 & 0 & 0\\
			0 & 1 & 0\\
			0 & 0 & 1 
		\end{array}
	\right)
	\left(
		\begin{array}{ccc}
			4 & -1 & 1\\
			4 & -8 & 1\\
			-2 & 1 & 5 
		\end{array}
	\right)\\
	& =
	\left(
		\begin{array}{ccc}
			1 & 0 & 0\\
			1 & 1 & 0\\
			-0.5 & 0 & 1 
		\end{array}
	\right)
	\left(
		\begin{array}{ccc}
			4 & -1 & 1\\
			0 & -7 & 0\\
			0 & 0.5 & 5.5 
		\end{array}
	\right)\\
	& =
	\left(
		\begin{array}{ccc}
			1 & 0 & 0\\
			1 & 1 & 0\\
			-0.5 & -1/14 & 1 
		\end{array}
	\right)
	\left(
		\begin{array}{ccc}
			4 & -1 & 1\\
			0 & -7 & 0\\
			0 & 0 & 5.5 
		\end{array}
	\right) \notag
	\end{aligned}
\end{equation}
首先利用前向替换法对方程组\textbf{LY=B}求解\textbf{Y}
\begin{equation}
	\begin{cases}
		y_1 = 7 \\
		y_1 + y_2 = -21 \\
		-\frac{1}{2}y_1 - \frac{1}{14}y_2 + y_3 = 15 \notag
	\end{cases}
\end{equation}
得到$\textbf{Y} = [7 \quad -28 \quad \frac{33}{2}]$.\\
接下来表示方程组\textbf{UX=Y}为
\begin{equation}
	\begin{cases}
		4x_1 - x_2 + x_3 = 7 \\
		-7x_2 = -28 \\
		5.5x_3 = \frac{33}{2} \notag
	\end{cases}
\end{equation}
得到$\textbf{X} = [2 \quad 4 \quad 3]$.\\
(2)方程组\ref{eq:1}的Jacobi迭代过程:\\
\[x_{k+1}=\frac{7+y_k-z_k}{4}\]
\[y_{k+1}=\frac{21+4x_k+z_k}{8}\]
\[z_{k+1}=\frac{15+2x_k-y_k}{5}\]
针对本题,我们可以简单编制如下程序求解:\\
\begin{lstlisting}[language=matlab]
syms x y z;
px(1)=1;
py(1)=2;
pz(1)=2;
format long;
for i=2:20
	px(i)=(7+py(i-1)-pz(i-1))/4
	py(i)=(21+4*px(i-1)+pz(i-1))/8
	pz(i)=(15+2*px(i-1)-py(i-1))/5
end
\end{lstlisting}
本题的迭代过程如表格\ref{tab:t1}所示.\\
\begin{table}[h]
	%\renewcommand{\arraystretch}{1.3}
	\caption{Jacobi迭代过程}
	\label{tab:t1}
	\centering
	\begin{tabular}{c|c|c|c}
		\hline
		k  &  $x_k$  & $y_k$  & $z_k$\\
		\hline
		0   & 1.0000000000 & 2.0000000000 & 2.0000000000 \\
		1	& 1.7500000000 & 3.3750000000 & 3.0000000000 \\
		2	& 1.8437500000 & 3.8750000000 & 3.0250000000 \\
		3	& 1.9625000000 & 3.9250000000 & 2.9625000000 \\
		4	& 1.9906250000 & 3.9765625000 & 3.0000000000 \\
		5	& 1.9941406250 & 3.9953125000 & 3.0009375000 \\
		6	& 1.9985937500 & 3.9971875000 & 2.9985937500 \\
		7	& 1.9996484375 & 3.9991210938 & 3.0000000000 \\
		8	& 1.9997802734 & 3.9998242188 & 3.0000351563 \\
		9	& 1.9999472656 & 3.9998945313 & 2.9999472656 \\
		\vdots & \vdots & \vdots & \vdots \\
		19  & 1.9999999993 & 3.9999999983 & 3.0000000018 \\
		20  & 2.00000000 & 4.00000000 & 3.00000000 \\
		\hline
	\end{tabular}
\end{table}

更一般的Jacobi迭代算法实现如下:\\
\begin{lstlisting}[language=matlab]
function X = jacobi(A, B, P ,delta, max1)
% Input - A is an NxN nonsingular matrix
%       - B is an Nx1 matrix
%       - P is an Nx1 matrix; the initial guess
%       - delta is the tolerance for P
%       - max1 is the maximum number of iterations
% Output - X is an N x 1 matrix: the jacobi approximation to
%           the solution of AX = B

N = length(B);
for k = 1:max1
	for j = 1:N
		X(j)=(B(j)-A(j,[1:j-1,j+1:N])*P([1:j-1,j+1:N]))/A(j,j);
	end
	err = abs (norm(X'-P));
	relerr = err/(norm(X) + eps);
	P = X';
	if (err < delta) || (relerr < delta)
		break,end
end
X=X';
end
\end{lstlisting}
方程组\ref{eq:1}的Gauss-Seidel迭代过程:\\
\[x_{k+1}=\frac{7+y_k-z_k}{4}\]
\[y_{k+1}=\frac{21+4x_{k+1}+z_k}{8}\]
\[z_{k+1}=\frac{15+2x_{k+1}-y_{k+1}}{5}\]
针对本题,我们可以简单编制如下程序求解:\\
\begin{lstlisting}[language=matlab]
syms x y z;
px(1)=1;
py(1)=2;
pz(1)=2;
format long;
for i=2:15
	px(i)=(7+py(i-1)-pz(i-1))/4
	py(i)=(21+4*px(i)+pz(i-1))/8
	pz(i)=(15+2*px(i)-py(i))/5
end
\end{lstlisting}
本题的迭代过程如表格\ref{tab:t2}所示.\\
\begin{table}[h]
	%\renewcommand{\arraystretch}{1.3}
	\caption{Gauss-Seidel迭代过程}
	\label{tab:t2}
	\centering
	\begin{tabular}{c|c|c|c}
		\hline
		k  &  $x_k$  & $y_k$  & $z_k$\\
		\hline
		0   & 1.0000000000 & 2.0000000000 & 2.0000000000 \\
		1	& 1.7500000000 & 3.3750000000 & 2.9500000000 \\
		2	& 1.9500000000 & 3.9687500000 & 2.9862500000 \\
		3	& 1.9956250000 & 3.9960937500 & 2.9990312500 \\
		4	& 1.9992656250 & 3.9995117188 & 2.9998039063 \\
		5	& 1.9999269531 & 3.9999389648 & 2.9999829883 \\
		6	& 1.9999889941 & 3.9999923706 & 2.9999971235 \\
		7	& 1.9999988118 & 3.9999990463 & 2.9999997154 \\
		8	& 1.9999998327 & 3.9999998808 & 2.9999999569 \\
		9	& 1.9999999809 & 3.9999999851 & 2.9999999954 \\
		\vdots & \vdots & \vdots & \vdots \\
		13  & 2.0000000000 & 4.0000000000 & 3.0000000000 \\
		\hline
	\end{tabular}
\end{table}

更一般的Jacobi迭代算法实现如下:
\begin{lstlisting}[language=matlab]
function X = gseid(A, B, P ,delta, max1)
% Input - A  is an NxN nonsingular matrix 
%       - B  is an Nx1 matrix
%       - P  is an Nx1 matrix; the initial guess
%       - delta is the tolerance for P
%       - max1 is the maximum number of iterations
% Output - X is an N x 1 matrix: the gauss-seidel
%           approximation to the solution of AX = B

N = length(B);
for k = 1:max1
	for j = 1:N
	if j == 1
		X(1) = (B(1)-A(1,2:N) *P(2:N))/A(1,1);
	elseif j == N
		X(N) = (B(N)-A(N,1:N-1)*(X(1:N-1))')/A(N,N);
	else
		% X contains the kth approximations and P the (k-1)st
		X(j) = (B(j)-A(j,1:j-1)*X(1:j-1)'...
				-A(j, j+1:N)*P(j+1:N))/A(j,j);
	end
end
err = abs(norm(X'-P));
relerr = err/(norm(X) + eps);
P = X';
if (err<de1ta) || (relerr<delta)
	break, end
end
X = X';
end
\end{lstlisting}


\end{document}
